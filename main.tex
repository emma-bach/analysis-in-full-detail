\iffalse
\DocumentMetadata{
	lang        = de,
	pdfstandard = ua-2,
	pdfstandard = a-4f, %or a-4
	tagging=on,
	tagging-setup={math/setup=mathml-SE} 
}
\fi
\documentclass{report}

% custom margins
\usepackage[a4paper,margin=1.5in]{geometry}
\renewcommand{\baselinestretch}{1.2}
%\AddToHook{cmd/section/before}{\clearpage}

% emma's long list of custom macros and universally used packages
%AMS packages - Symbols, "Theorem" and "Proof" Environments
\usepackage{amsmath}
\usepackage{amssymb}
\usepackage{amsthm}
\usepackage{physics}

% Nicer Headers and Footers
\usepackage{fancyhdr}

\renewcommand{\chaptermark}[1]{
	\markboth{\chaptername\ \thechapter:\ #1}
	{}
}

\renewcommand{\sectionmark}[1]{
	\markright{\thesection{}\ -- \ #1}
	{}
}

\fancyhead[L]{\leftmark\\\rightmark}
\fancyhead[R]{}

\newcommand*\parttitle{}
\let\origpart\part
\renewcommand*{\part}[2][]{ %
	\ifx\\#1\\% optional argument not present?
	\origpart{#2}%
	\renewcommand*\parttitle{#2}%
	\else
	\origpart[#1]{#2}%
	\renewcommand*\parttitle{#1}%
	\fi
}
\usepackage[yyyymmdd]{datetime}
\renewcommand{\dateseparator}{-}

% Table of Contents
\usepackage[titles]{tocloft}
\usepackage[titletoc]{appendix}

% Tikz and Graphics
\usepackage{tikz}
\usepackage{tikz-cd}
\usepackage{xcolor}
\usepackage[many]{tcolorbox}

% Nicer Underlining
\usepackage{contour}
\usepackage{ulem}

% Multiple Text Columns
\usepackage{multicol}

% Margin Notes
\usepackage{sidenotes}

% FloatBarrier
\usepackage{placeins} 

% hyperref should be last apparently
\usepackage{hyperref}

% nicer text
\usepackage{microtype} % general improvements to text appearance
\usepackage[T2A,T1]{fontenc}
\usepackage[utf8]{inputenc}
\usepackage[osf, helvratio=.9]{newpxtext} % text font
\usepackage{newpxmath} % math font


%--- RENEWED COMMANDS ---

% Variant Greek Letters
\renewcommand\epsilon{\varepsilon}
\renewcommand\phi{\varphi}

\DeclareMathAlphabet{\mathbbold}{U}{bbold}{m}{n}
\newcommand*{\boldone}{\mathbbold{1}}

% Nicer Table of Contents
\renewcommand\cftsecdotsep{\cftdot}
\renewcommand\cftsubsecdotsep{\cftdot}

%--- TEXT FORMATING ---

% nice underlining
\renewcommand{\ULdepth}{1.6pt}
\contourlength{0.8pt}
\newcommand{\ul}[1]{%
	\uline{\phantom{#1}}%
	\llap{\contour{white}{#1}}%
}

% shorthand
\newcommand{\tbf}[1]{\textbf{#1}}
\newcommand{\tn}[1]{\textnormal{#1}}
\newcommand{\tit}[1]{\textit{#1}}
\newcommand{\ttt}[1]{\texttt{#1}}

% make \textbf automatically include math
\let\oldtextbf=\textbf
\renewcommand\textbf[1]{{\boldmath\oldtextbf{#1}}}

\newcommand{\mc}[1]{\mathcal{#1}}

\newcommand{\ol}[1]{\overline{{#1}}}

% Underlined, bold, non-cursive Theorem Name
\newcommand{\theoremname}[1]{\textnormal{\tbf{(#1):}}}

% Starts a new paragraph without indentation
% and with an empty line between paragraphs
\newcommand*{\newpar}{\par\vspace{\baselineskip}\noindent}

% Make inf height match sup height
\renewcommand{\inf}{\mathop{\mathrm{inf}\vphantom{\mathrm{sup}}}}

% Make tildes more readable
\renewcommand{\tilde}{\widetilde}

%--- RELATION SYMBOLS AND OPERATORS ---
\newcommand{\surj}{\twoheadrightarrow}
\newcommand{\inj}{\hookrightarrow}
\newcommand{\iso}{\overset{\sim}{\rightarrow}}
\newcommand{\symdiff}{\vartriangle}
\newcommand{\trans}{\twoheadrightarrow}
\newcommand{\tensor}{\otimes}
\newcommand{\Tensor}{\bigotimes}
\renewcommand{\mid}{~\middle|~}

%--- STUFF IN FANCY BRACKETS ---
\newcommand{\stack}[2]{\begin{array}{c} #1\\#2\end{array}}
\newcommand{\trinom}[3]{\begin{pmatrix} #1\\#2\\#3\end{pmatrix}}
\newcommand{\scalar}[2]{\left\langle #1, #2 \right\rangle}
\newcommand{\angles}[1]{\left\langle #1 \right\rangle}
\newcommand{\lr}{\qty}


%--- LETTERS ---
\newcommand{\dmu}{\ d\mu}
\newcommand{\dist}{\textnormal{dist}}
\newcommand{\spt}{\textnormal{spt}}

% mathbb
\newcommand{\bC}{\mathbb{C}}
\newcommand{\bF}{\mathbb{F}}
\newcommand{\bN}{\mathbb{N}}
\newcommand{\bQ}{\mathbb{Q}}
\newcommand{\bR}{\mathbb{R}}
\newcommand{\bZ}{\mathbb{Z}}

% mathcal
\newcommand{\cA}{\mathcal{A}}
\newcommand{\cB}{\mathcal{B}}
\newcommand{\cC}{\mathcal{C}}
\newcommand{\cD}{\mathcal{D}}
\newcommand{\cE}{\mathcal{E}}
\newcommand{\cF}{\mathcal{F}}
\newcommand{\cH}{\mathcal{H}}
\newcommand{\cI}{\mathcal{I}}
\newcommand{\cL}{\mathcal{L}}
\newcommand{\cM}{\mathcal{M}}
\newcommand{\cN}{\mathcal{N}}
\newcommand{\cO}{\mathcal{O}}
\newcommand{\cP}{\mathcal{P}}
\newcommand{\cQ}{\mathcal{Q}}
\newcommand{\cR}{\mathcal{R}}
\newcommand{\cS}{\mathcal{S}}
\newcommand{\cT}{\mathcal{T}}
\newcommand{\cW}{\mathcal{W}}
\newcommand{\cX}{\mathcal{X}}
\newcommand{\cZ}{\mathcal{Z}}

%mathfrac
\newcommand{\fa}{\mathfrak{a}}

% real numbers in fancy costumes
\newcommand{\barR}{\ol{\mathbb{R}}}
\newcommand{\bRnn}{\mathbb{R}^{n \times n}}
\newcommand{\bRmn}{\mathbb{R}^{m \times n}}
\newcommand{\bRnm}{\mathbb{R}^{n \times m}}

% vectors
\newcommand{\vzero}{\kern-1.2pt\vec{\kern1.2pt 0}} 
\renewcommand{\va}{\vec{a}}
\renewcommand{\vb}{\vec{b}}
\newcommand{\vc}{\vec{c}}
\newcommand{\ve}{\vec{e}}
\newcommand{\vF}{\vec{F}}
\newcommand{\vh}{\vec{h}}
\newcommand{\vp}{\vec{p}}
\newcommand{\vr}{\vec{r}}
\newcommand{\vs}{\vec{s}}
\newcommand{\vT}{\vec{T}}
\renewcommand{\vu}{\vec{u}}
\renewcommand{\vv}{\vec{v}}
\newcommand{\vw}{\vec{w}}
\newcommand{\vW}{\vec{W}}
\newcommand{\vx}{\vec{x}}
\newcommand{\vy}{\vec{y}}
\newcommand{\vz}{\vec{z}}

\newcommand{\veta}{\vec{\eta}}

\renewcommand{\grad}{\vec{\nabla}}

% bold letters
\newcommand{\tbA}{\mathbf{A}}
\newcommand{\tbB}{\mathbf{B}}
\newcommand{\tbC}{\mathbf{C}}
\newcommand{\tbD}{\mathbf{D}}
\newcommand{\tbE}{\mathbf{E}}
\newcommand{\tbY}{\mathbf{Y}}
\newcommand{\tbZ}{\mathbf{Z}}

% sequences
\newcommand{\an}{(a_n)_{n \in \bN}}
\newcommand{\bn}{(b_n)_{n \in \bN}}
\newcommand{\sn}{(s_n)_{n \in \bN}}
\newcommand{\iinI}{_{i \in I}}
\newcommand{\iinN}{_{i \in \bN}}

% special functions
\newcommand{\at}{\textnormal{at}}
\newcommand{\ggT}{\textnormal{ggT}}
\newcommand{\kgV}{\textnormal{kgV}}
\newcommand{\id}{\textnormal{id}}
\newcommand{\Id}{\textnormal{Id}}
\newcommand{\im}{\textnormal{im}}
\newcommand{\inv}{\textnormal{inv}}
\newcommand{\ord}{\textnormal{ord}\ }
\newcommand{\rang}{\textnormal{rang}\ }
\renewcommand{\tr}{\textnormal{tr}\ }
\newcommand{\vol}{\textnormal{vol}}
\newcommand{\cond}{\textnormal{cond}}
\newcommand{\sgn}{\textnormal{sgn}}
\renewcommand{\char}{\textnormal{char}}
\newcommand{\Frac}{\textnormal{Frac}}
\newcommand{\card}{\textnormal{card}}

% induced set systems
\newcommand{\sigalg}[1]{\angles{#1}_\sigma}
\newcommand{\boolalg}[1]{\angles{#1}_\cB}
\newcommand{\mono}[1]{\angles{#1}_\cM}
\newcommand{\monotone}[1]{\angles{#1}_\cM}

% groups and sets
\newcommand{\GL}{\text{GL}}
\newcommand{\SO}{\text{SO}}
\newcommand{\Hess}[1]{\text{Hess}(#1)}
\renewcommand{\div}{\tn{div}}

\newcommand{\Grp}{\textnormal{Grp}}
\newcommand{\Mag}{\textnormal{Mag}}
\newcommand{\Mon}{\textnormal{Mon}}
\newcommand{\Ens}{\textnormal{Ens}}
\newcommand{\Hom}{\textnormal{Hom}}
\newcommand{\Aut}{\textnormal{Aut}}

\newcommand{\Mat}{\textnormal{Mat}}
\newcommand{\Matnn}{\textnormal{Mat}(n \times n)}
\newcommand{\MatBB}[3]{\textnormal{Mat}^{#1}_{#2}\left(#3\right)}

% --- THEOREM AND PROOF TYPES ---
% \newtheorem{codename}{printedname}[countedwith]
\newtheorem{lemma}{Lemma}[section]
\newtheorem{theorem}[lemma]{Theorem}
\newtheorem{proposition}[lemma]{Proposition}
\newtheorem{corollary}[lemma]{Corollary}

\theoremstyle{definition}
\newtheorem{definition}[lemma]{Definition}
\newtheorem{example}[lemma]{Example}
\newtheorem{counterexample}[lemma]{Counterexample}
\newtheorem{observation}[lemma]{Observation}
\newtheorem{anmerkung}[lemma]{Comment}
\newtheorem{question}[lemma]{Question}
\newtheorem{application}[lemma]{Application}
\newtheorem{reminder}[lemma]{Reminder}
\newtheorem{consequence}[lemma]{Consequence}
\newtheorem{summary}[lemma]{Summary}

% --- CUSTOM ENVIRONMENTS ---

% sketchy proofs marked with untrustworthy QED symbol
\newenvironment{proofsketch}{\begin{proof}[Beweisskizze]\renewcommand*{\qedsymbol}{\("\square"\)}}{\end{proof}}

% align with exactly one number for labeling
\NewDocumentEnvironment{nalign}{}{\equation\aligned}{\endaligned\endequation}

% colored box behind proofs
\tcolorboxenvironment{proof}{
	colback=white,
	boxrule=0pt,
	frame hidden,
	borderline west={1pt}{0pt}{black},
	before skip=0.75cm,
	after skip=0.75cm,
	sharp corners,
	breakable,
	enhanced,
}

\renewenvironment{proofsketch}{\begin{proof}[Proof Sketch]\renewcommand*{\qedsymbol}{\("\square"\)}}{\end{proof}}

\pagestyle{fancy} %allows headers

\lhead{Emma Yoneda}
\rhead{\today}

\begin{document}
	
	\begin{titlepage}
	\centering
	{\Huge\bfseries Analysis for people who don't like skipping details\par}
    \vspace{0.5cm}
	{\Large\itshape Emma Yoneda\par}
	\vfill
	

% Bottom of the page
	{\large \today\par}
\end{titlepage}

	\tableofcontents
	\thispagestyle{fancy}
	\chapter{Sets and Orders}
	\chapter{Ordered Fields}
		\iffalse
		\begin{definition}
			An \tbf{ordered commutative Ring} $(R,P)$ is a commutative Ring $K$ equipped with a set $P$ of \tbf{positive Elements}, such that:
			\begin{enumerate}
				\item For every $x \in K$, we have exactly one of $x \in P$ ($x$ is \tit{positive}), $x = 0$, or $-x \in P$ ($x$ is \tit{negative}).
				\item If $x$ and $y$ are positive, then so are $x + y$ and $x \cdot y$.
			\end{enumerate}
		\end{definition}
		We say that $K$ is \tit{ordered by $P$}.
		\fi
		\begin{definition}
		An \tbf{ordered commutative Ring} $(R, \leq)$ is a commutative Ring $R$ equipped with an \tbf{ordering relation} $\leq$, such that for all $a,b,c \in R$, we have:
		\begin{enumerate}
			\item $\leq$ defines a \tit{total order} on $F$. i.e:
			\begin{enumerate}
				\item $a \leq a$ (the order is \tit{reflexive}),
				\item $a \leq b \wedge b \leq c \implies a \leq c$ (the order is \tit{transitive}),
				\item $a \leq b \wedge b \leq a \implies a = b$ (the order is \tit{antisymmetric}),
				\item $a \leq b \vee b \leq a$ (the order is \tit{strongly connected})
			\end{enumerate}
			\item $a \leq b \implies a + c \leq b + c$
			\item $0 \leq a \wedge 0 \leq b \implies 0 \leq ab$
		\end{enumerate}
		\end{definition}
				\begin{lemma}
			For every ordered commutative Ring $(R, \leq)$ and $a \in R$, we have $-a \leq 0 \leq a$ or $a \leq 0 \leq -a$.
		\end{lemma}
		\begin{proof}
			Since the order $\leq$ is strongly connected, we have $a \leq 0$ or $0 \leq a$.
			\begin{enumerate}
				\item If $a \leq 0$, then we have $-a + a \leq 0 + -a$, i.e. $a \leq 0 \leq -a$,
				\item if $0 \leq a$, then we have $-a + 0 \leq -a + a$, i.e. $-a \leq 0 \leq a$
			\end{enumerate}
		\end{proof}
		\begin{lemma}
			Let $a \in (R, \leq)$. Then $0 \leq a^2$.
		\end{lemma}
		\begin{proof}
			Since the order $\leq$ is strongly connected, we have $a \leq 0$ or $0 \leq a$.
			\begin{enumerate}
				\item If $0 \leq a$, then we have $0 \leq a \cdot a = a^2$,
				\item if $a \leq 0$, then $0 \leq -a$ and we have $0 \leq -a \cdot (-a) = a^2$.
			\end{enumerate}
			 
		\end{proof}
		\begin{lemma}
			Every ordered commutative Ring has characteristic $0$.
		\end{lemma}
		\begin{proof}
			Assume that $F$ is a field of characteristic $p$. Then an ordering relation would need to fulfill:
			\begin{align*}
				1 \leq 1 + 1 \leq \sum_{i = 1}^p 1 = 0 \leq 1
			\end{align*}
		\end{proof}
	\section{The Archimedean Property}
	\section{On the Importance of the Real Numbers}
		\begin{theorem}
			Let $F$ be an arbitrary archimedean ordered field. Then $F$ is isomorphic to a subfield of the real number $\bR$.
		\end{theorem}
		\begin{theorem}
			Let $F$ be an arbitrary ordered field. Then $F$ has the least-upper-bound property if and only if it is archimedean and cauchy complete.
		\end{theorem}
	\chapter{Topology}
		\section{Topological Spaces}
		\section{Metric Spaces}
	\chapter{Topological Vector Spaces}
		\section{Topological Vector Spaces}
		\section{Normed Vector Spaces}
		\section{Banach Spaces}
	\chapter{Differentiation}
		\section{Frechét Spaces}
		\section{The Gateaux Derivative}
		\section{The Frechét Derivative}
	\chapter{Measure Theory}
		\section{Set Algebras}
		\section{Measure Spaces}
		\section{The Lebesque Measure}
	\chapter{Integration}
		\section{The Bochner Integral}
		
\end{document}